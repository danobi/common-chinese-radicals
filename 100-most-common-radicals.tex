% !TEX encoding = UTF-8 Unicode
% !TEX TS-program = XeLaTeX
\documentclass[fontsize=12pt,
			 parskip=full,
			 a4paper,
			 landscape,
			 pagesize,
			 DIV=12]{scrreprt}

\usepackage{xltxtra}
\usepackage{fontspec}
\usepackage{xcolor}
\usepackage{xeCJK}
\usepackage{zhnumber}
\usepackage{array}
\usepackage{longtable}
\usepackage[colorlinks]{hyperref}
\usepackage{catchfile}
\usepackage{collcell}

\setmainfont{Myriad Pro}
\setsansfont{Myriad Pro}
\setCJKmainfont[BoldFont={Kaiti SC Bold}]{Kaiti SC}

\newcommand{\nl}{\\}

\newcommand{\lnk}[2]{\href{http://www.archchinese.com/chinese_english_dictionary.html?find=#1}{#2}}

\newcommand*{\clnk}[1]{%
\IfFileExists{./#1.dat}{}{%
\immediate\write18{echo #1 | perl urlenc.pl -l > #1.dat}}%
\CatchFileDef\encoded{#1.dat}{\endlinechar=-1}%
\href{http://www.archchinese.com/chinese_english_dictionary.html?find=\encoded}{#1}}

\renewcommand{\arraystretch}{1.4}

\pagestyle{myheadings}
\markright{\normalsize\bfseries The 100 Most Common Chinese Radicals 部首}

\begin{document}
\begin{longtable}{|>{\large}c|>{\large}c|>{\large}c|p{3cm}|p{1.5cm}|>{\large}l|p{7cm}|p{2.5cm}|}
\hline
\normalsize\bfseries SC & \normalsize\bfseries TC & \normalsize\bfseries Variant & \bfseries Meaning & \bfseries Pinyin & \normalsize\bfseries Examples & \bfseries Comments & \bfseries Colloquial \\ \hline
\endhead
\hline
\endfoot
% !TEX encoding = UTF-8 Unicode
% !TEX TS-program = XeLaTeX
\documentclass[fontsize=12pt,
			 parskip=full,
			 a4paper,
			 landscape,
			 pagesize,
			 DIV=12]{scrreprt}

\usepackage{xltxtra}
\usepackage{fontspec}
\usepackage{xcolor}
\usepackage{xeCJK}
\usepackage{zhnumber}
\usepackage{array}
\usepackage{longtable}
\usepackage[colorlinks]{hyperref}
\usepackage{catchfile}
\usepackage{collcell}

\setmainfont{Myriad Pro}
\setsansfont{Myriad Pro}
\setCJKmainfont[BoldFont={Kaiti SC Bold}]{Kaiti SC}

\newcommand{\nl}{\\}

\newcommand{\lnk}[2]{\href{http://www.archchinese.com/chinese_english_dictionary.html?find=#1}{#2}}

\newcommand*{\clnk}[1]{%
\immediate\write18{echo #1 | perl urlenc.pl -l > #1.dat}%
\CatchFileDef\encoded{#1.dat}{\endlinechar=-1}%
\href{http://www.archchinese.com/chinese_english_dictionary.html?find=\encoded}{#1}}

\renewcommand{\arraystretch}{1.4}

\pagestyle{myheadings}
\markright{\normalsize\bfseries The 100 Most Common Chinese Radicals 部首}

\begin{document}
\begin{longtable}{|>{\large}c|>{\large}c|>{\large}c|p{3cm}|p{1.5cm}|>{\large}l|p{7cm}|p{2.5cm}|}
\hline
\normalsize\bfseries SC & \normalsize\bfseries TC & \normalsize\bfseries Variant & \bfseries Meaning & \bfseries Pinyin & \normalsize\bfseries Examples & \bfseries Comments & \bfseries Colloquial \\ \hline
\endhead
\hline
\endfoot
% !TEX encoding = UTF-8 Unicode
% !TEX TS-program = XeLaTeX
\documentclass[fontsize=12pt,
			 parskip=full,
			 a4paper,
			 landscape,
			 pagesize,
			 DIV=12]{scrreprt}

\usepackage{xltxtra}
\usepackage{fontspec}
\usepackage{xcolor}
\usepackage{xeCJK}
\usepackage{zhnumber}
\usepackage{array}
\usepackage{longtable}
\usepackage[colorlinks]{hyperref}
\usepackage{catchfile}
\usepackage{collcell}

\setmainfont{Myriad Pro}
\setsansfont{Myriad Pro}
\setCJKmainfont[BoldFont={Kaiti SC Bold}]{Kaiti SC}

\newcommand{\nl}{\\}

\newcommand{\lnk}[2]{\href{http://www.archchinese.com/chinese_english_dictionary.html?find=#1}{#2}}

\newcommand*{\clnk}[1]{%
\immediate\write18{echo #1 | perl urlenc.pl -l > #1.dat}%
\CatchFileDef\encoded{#1.dat}{\endlinechar=-1}%
\href{http://www.archchinese.com/chinese_english_dictionary.html?find=\encoded}{#1}}

\renewcommand{\arraystretch}{1.4}

\pagestyle{myheadings}
\markright{\normalsize\bfseries The 100 Most Common Chinese Radicals 部首}

\begin{document}
\begin{longtable}{|>{\large}c|>{\large}c|>{\large}c|p{3cm}|p{1.5cm}|>{\large}l|p{7cm}|p{2.5cm}|}
\hline
\normalsize\bfseries SC & \normalsize\bfseries TC & \normalsize\bfseries Variant & \bfseries Meaning & \bfseries Pinyin & \normalsize\bfseries Examples & \bfseries Comments & \bfseries Colloquial \\ \hline
\endhead
\hline
\endfoot
% !TEX encoding = UTF-8 Unicode
% !TEX TS-program = XeLaTeX
\documentclass[fontsize=12pt,
			 parskip=full,
			 a4paper,
			 landscape,
			 pagesize,
			 DIV=12]{scrreprt}

\usepackage{xltxtra}
\usepackage{fontspec}
\usepackage{xcolor}
\usepackage{xeCJK}
\usepackage{zhnumber}
\usepackage{array}
\usepackage{longtable}
\usepackage[colorlinks]{hyperref}
\usepackage{catchfile}
\usepackage{collcell}

\setmainfont{Myriad Pro}
\setsansfont{Myriad Pro}
\setCJKmainfont[BoldFont={Kaiti SC Bold}]{Kaiti SC}

\newcommand{\nl}{\\}

\newcommand{\lnk}[2]{\href{http://www.archchinese.com/chinese_english_dictionary.html?find=#1}{#2}}

\newcommand*{\clnk}[1]{%
\immediate\write18{echo #1 | perl urlenc.pl -l > #1.dat}%
\CatchFileDef\encoded{#1.dat}{\endlinechar=-1}%
\href{http://www.archchinese.com/chinese_english_dictionary.html?find=\encoded}{#1}}

\renewcommand{\arraystretch}{1.4}

\pagestyle{myheadings}
\markright{\normalsize\bfseries The 100 Most Common Chinese Radicals 部首}

\begin{document}
\begin{longtable}{|>{\large}c|>{\large}c|>{\large}c|p{3cm}|p{1.5cm}|>{\large}l|p{7cm}|p{2.5cm}|}
\hline
\normalsize\bfseries SC & \normalsize\bfseries TC & \normalsize\bfseries Variant & \bfseries Meaning & \bfseries Pinyin & \normalsize\bfseries Examples & \bfseries Comments & \bfseries Colloquial \\ \hline
\endhead
\hline
\endfoot
\input{100-most-common-radicals.inc}
\end{longtable}
{\em Source:} \url{http://www.hackingchinese.com/kickstart-your-character-learning-with-the-100-most-common-radicals/}

\end{document}
\end{longtable}
{\em Source:} \url{http://www.hackingchinese.com/kickstart-your-character-learning-with-the-100-most-common-radicals/}

\end{document}
\end{longtable}
{\em Source:} \url{http://www.hackingchinese.com/kickstart-your-character-learning-with-the-100-most-common-radicals/}

\end{document}
\end{longtable}
{\em Source:} \url{http://www.hackingchinese.com/kickstart-your-character-learning-with-the-100-most-common-radicals/}

{\em Typeset by} Markus Ackermann, \url{http://github.com/saigyo}, {\em using} \XeLaTeX.

\end{document}